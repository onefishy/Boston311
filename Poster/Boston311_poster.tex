\documentclass[landscape,a0paper,fontscale=0.292]{baposter}

\usepackage[vlined]{algorithm2e}
\usepackage{times}
\usepackage{calc}
\usepackage{url}
\usepackage{graphicx}
\usepackage{amsmath}
\usepackage{amssymb}
\usepackage{relsize}
\usepackage{multirow}
\usepackage{booktabs}

\usepackage{graphicx}
\usepackage{multicol}
\usepackage[T1]{fontenc}
\usepackage{ae}

\graphicspath{{images/}}

 %%%%%%%%%%%%%%%%%%%%%%%%%%%%%%%%%%%%%%%%%%%%%%%%%%%%%%%%%%%%%%%%%%%%%%%%%%%%%%%%
 %%%% Some math symbols used in the text
 %%%%%%%%%%%%%%%%%%%%%%%%%%%%%%%%%%%%%%%%%%%%%%%%%%%%%%%%%%%%%%%%%%%%%%%%%%%%%%%%
 % Format 
 \newcommand{\RotUP}[1]{\begin{sideways}#1\end{sideways}}


 %%%%%%%%%%%%%%%%%%%%%%%%%%%%%%%%%%%%%%%%%%%%%%%%%%%%%%%%%%%%%%%%%%%%%%%%%%%%%%%%
 % Multicol Settings
 %%%%%%%%%%%%%%%%%%%%%%%%%%%%%%%%%%%%%%%%%%%%%%%%%%%%%%%%%%%%%%%%%%%%%%%%%%%%%%%%
 \setlength{\columnsep}{0.7em}
 \setlength{\columnseprule}{0mm}


 %%%%%%%%%%%%%%%%%%%%%%%%%%%%%%%%%%%%%%%%%%%%%%%%%%%%%%%%%%%%%%%%%%%%%%%%%%%%%%%%
 % Save space in lists. Use this after the opening of the list
 %%%%%%%%%%%%%%%%%%%%%%%%%%%%%%%%%%%%%%%%%%%%%%%%%%%%%%%%%%%%%%%%%%%%%%%%%%%%%%%%
 \newcommand{\compresslist}{%
 \setlength{\itemsep}{1pt}%
 \setlength{\parskip}{0pt}%
 \setlength{\parsep}{0pt}%
 }


 %%%%%%%%%%%%%%%%%%%%%%%%%%%%%%%%%%%%%%%%%%%%%%%%%%%%%%%%%%%%%%%%%%%%%%%%%%%%%%
 % Formating
 \newcommand{\Matrix}[1]{\begin{bmatrix} #1 \end{bmatrix}}
 \newcommand{\Vector}[1]{\begin{pmatrix} #1 \end{pmatrix}}

 \newcommand*{\norm}[1]{\mathopen\| #1 \mathclose\|}% use instead of $\|x\|$
 \newcommand*{\abs}[1]{\mathopen| #1 \mathclose|}% use instead of $\|x\|$
 \newcommand*{\normLR}[1]{\left\| #1 \right\|}% use instead of $\|x\|$

 \newcommand*{\SET}[1]  {\ensuremath{\mathcal{#1}}}
 \newcommand*{\FUN}[1]  {\ensuremath{\mathcal{#1}}}
 \newcommand*{\MAT}[1]  {\ensuremath{\boldsymbol{#1}}}
 \newcommand*{\VEC}[1]  {\ensuremath{\boldsymbol{#1}}}
 \newcommand*{\CONST}[1]{\ensuremath{\mathit{#1}}}

 \DeclareMathOperator*{\argmax}{arg\,max}
 \DeclareMathOperator*{\diag}{diag}
 \DeclareMathOperator*{\argmin}{arg\,min}
 \DeclareMathOperator*{\vectorize}{vec}
 \DeclareMathOperator*{\reshape}{reshape}

 %-----------------------------------------------------------------------------
 % Differentiation
 \newcommand*{\Nabla}[1]{\nabla_{\!#1}}

 \renewcommand*{\d}{\mathrm{d}}
 \newcommand*{\dd}{\partial}

 \newcommand*{\At}[2]{\ensuremath{\left.#1\right|_{#2}}}
 \newcommand*{\AtZero}[1]{\At{#1}{\pp=\VEC 0}}

 \newcommand*{\diffp}[2]{\ensuremath{\frac{\dd #1}{\dd #2}}}
 \newcommand*{\diffpp}[3]{\ensuremath{\frac{\dd^2 #1}{\dd #2 \dd #3}}}
 \newcommand*{\diffppp}[4]{\ensuremath{\frac{\dd^3 #1}{\dd #2 \dd #3 \dd #4}}}
 \newcommand*{\difff}[2]{\ensuremath{\frac{\d #1}{\d #2}}}
 \newcommand*{\diffff}[3]{\ensuremath{\frac{\d^2 #1}{\d #2 \d #3}}}
 \newcommand*{\difffp}[3]{\ensuremath{\frac{\dd\d #1}{\d #2 \dd #3}}}
 \newcommand*{\difffpp}[4]{\ensuremath{\frac{\dd^2\d #1}{\d #2 \dd #3 \dd #4}}}

 \newcommand*{\diffpAtZero}[2]{\ensuremath{\AtZero{\diffp{#1}{#2}}}}
 \newcommand*{\diffppAtZero}[3]{\ensuremath{\AtZero{\diffpp{#1}{#2}{#3}}}}
 \newcommand*{\difffAt}[3]{\ensuremath{\At{\difff{#1}{#2}}{#3}}}
 \newcommand*{\difffAtZero}[2]{\ensuremath{\AtZero{\difff{#1}{#2}}}}
 \newcommand*{\difffpAtZero}[3]{\ensuremath{\AtZero{\difffp{#1}{#2}{#3}}}}
 \newcommand*{\difffppAtZero}[4]{\ensuremath{\AtZero{\difffpp{#1}{#2}{#3}{#4}}}}

 %-----------------------------------------------------------------------------
 % Defined
 % How should the defined operator look like (:= or ^= ==)
 % (I want back my :=, it is so much better than ^= because (1) it has a
 % direction and (2) everyone here uses it.)
 %
 % Use :=
 %\newcommand*{\defined}{\ensuremath{\mathrel{\mathop{:}}=}}
 %\newcommand*{\definedRight}{\ensuremath{=\mathrel{\mathop{:}}}}
 % Use ^=
 \newcommand*{\defined}{\ensuremath{\triangleq}}
 \newcommand*{\definedRight}{\ensuremath{\triangleq}}
 % Use = with three bars
 %\newcommand*{\defined}{\ensuremath{?}}
 %\newcommand*{\definedRight}{\ensuremath{?}}

 %%%%%%%%%%%%%%%%%%%%%%%%%%%%%%%%%%%%%%%%%%%%%%%%%%%%%%%%%%%%%%%%%%%%%%%%%%%%%%
 % Symbols used in the paper

 %-----------------------------------------------------------------------------
 % The Methods
 \newcommand*{\ICIA}{\emph{ICIA}}
 \newcommand*{\CoDe}{\emph{CoDe}}
 \newcommand*{\LinCoDe}{\emph{LinCoDe}}
 \newcommand*{\CoNe}{\emph{CoNe}}
 \newcommand*{\CoLiNe}{\emph{CoLiNe}}
 \newcommand*{\LinCoLiNe}{\emph{LinCoLiNe}}
\usepackage[font=small,labelfont=bf]{caption} 
 % inter eye distance
 \newcommand*{\ied}{IED}

 %-----------------------------------------------------------------------------
 % Koerper
 %%\newcommand*{\RR}{\mathbb{R}}
 %\newcommand*{\RR}{{I\hspace{-3.5pt}R}}
 %\newcommand*{\RR}{{\mathrm{I\hspace{-2.7pt}R}}}

 \font\dsfnt=dsrom12

 \DeclareSymbolFont{nark}{U}{dsrom}{m}{n}
 \DeclareMathSymbol{\NN}{\dsfnt}{nark}{`N}
 \DeclareMathSymbol{\RR}{\dsfnt}{nark}{`R}
 \DeclareMathSymbol{\ZZ}{\dsfnt}{nark}{`Z}

 %-----------------------------------------------------------------------------
 % Domains
 \newcommand*{\D}{\mathcal{D}}
 \newcommand*{\I}{\mathcal{I}}

 %-----------------------------------------------------------------------------
 % Texture coordinates
 \newcommand*{\rr}{\VEC{r}}

 %-----------------------------------------------------------------------------
 % Parameters
 \newcommand*{\pt}{\VEC{\tau}}
 \newcommand*{\pr}{\VEC{\rho}}
 \newcommand*{\pp}{\VEC{p}}
 \newcommand*{\qq}{\VEC{q}}
 \newcommand*{\xx}{\VEC{x}}
 \newcommand*{\deltaq}{\Delta \qq}
 \newcommand*{\deltap}{\Delta \pp}
 \newcommand*{\zz}{\VEC{z}}
 \newcommand*{\pa}{\VEC{\alpha}}
 \newcommand*{\qa}{\VEC{\alpha}}
 \newcommand*{\pb}{\VEC{\beta}}

 %-----------------------------------------------------------------------------
 % Optimal appearance parameters
 \newcommand*{\pbh}[1]{\ensuremath{\hat{\pb}({#1})}}

 %-----------------------------------------------------------------------------
 % Warp basis
 \newcommand*{\M}[1]{\ensuremath{M({#1})}}
 \newcommand*{\LL}[1]{\ensuremath{L({#1})}}

 %-----------------------------------------------------------------------------
 % Matrices of the texture model
 \newcommand*{\AM}[1]{\ensuremath{\Lambda(#1)}}               % Lambda(beta) 
 \newcommand*{\AMr}[2]{\ensuremath{\Lambda(#1; #2)}}        % Lambda(r, beta)

 \newcommand*{\As}{A}         % Continuous Basis symbol
 \newcommand*{\afs}{a}        % Continuous mean symbol
 \newcommand*{\A}[1]{\As(#1)}         % Continuous Basis
 \newcommand*{\af}[1]{\afs(#1)}        % Continuous mean

\usepackage{tabularx}
 %-----------------------------------------------------------------------------
 % Matrices of the shape model
 \newcommand*{\MU}{\VEC{\mu}}
 \newcommand*{\MM}{\MAT{M}}

 %-----------------------------------------------------------------------------
 %% The project out matrix and operator
 \newcommand*{\INT}{\MAT{P}}
 \newcommand*{\INTf}{P}

 %-----------------------------------------------------------------------------
 % The identity matrix
 \newcommand*{\EYEtwo}{\Matrix{1 & 0\\0&1}}
 \newcommand*{\EYE}{\MAT E}
 \newcommand*{\EYEf}{E}

 % Wether to use subscripts or brackets for some function arguments
 % can be decided by commenting out the corresponding functions underneath
 %-----------------------------------------------------------------------------
 % Mapping
 \newcommand*{\Cs}[1]{\ensuremath{C^{#1}}} % C symbol
 \newcommand*{\C}[2]{\ensuremath{C^{#1}(#2)}} % Use C with brackets

 %-----------------------------------------------------------------------------
 % Objective function
 \newcommand*{\Fs}{\ensuremath{F}}              % F symbol
 \newcommand*{\F}[1]{\ensuremath{\Fs(#1)}}       % Use F with brackets    F(q)

 %-----------------------------------------------------------------------------
 % Approximated objective functions
 \newcommand*{\FFs}{\tilde{F}}                     % ~F symbol
 \newcommand*{\FF}[1]{\ensuremath{\FFs(#1)}}       % Use ~F with brackets    F(q)

 %-----------------------------------------------------------------------------
 % residual function
 \newcommand*{\es}{\ensuremath{f}}              % R symbol

 \newcommand*{\e}[1]{\ensuremath{\es(#1)}}         % R(q)
 \newcommand*{\er}[2]{\ensuremath{\es(#1; #2)}}    % R(r; q)

 %-----------------------------------------------------------------------------
 % Approximated residual functions
 \newcommand*{\ees}{\tilde{f}}                       % ~R symbol
 \newcommand*{\ee}[1]{\ensuremath{\ees(#1)}}       % ~R(q)
 \newcommand*{\eer}[2]{\ensuremath{\ees(#2; #1)}}  % ~R(r; q)

 %-----------------------------------------------------------------------------
 % Warps
 \newcommand*{\Vs}{\ensuremath{V}}
 \newcommand*{\VLins}{\ensuremath{\Vs^{\text{Ortho}}}}
 \newcommand{\VModels}{\ensuremath{\Vs^{\text{Model}}}}
 \newcommand*{\Ws}{\ensuremath{W}}

 \newcommand{\V}[1]{\ensuremath{\Vs(#1)}}
 \newcommand{\VModel}[1]{\ensuremath{\VModels(#1)}}
 \newcommand{\Vr}[2]{\ensuremath{\Vs(#1; #2)}}
 \newcommand{\VInvr}[2]{\ensuremath{\Vs^{-1}(#1; #2)}}
 \newcommand{\VrLin}[2]{\ensuremath{\VLins(#1; #2)}}
 \newcommand{\W}[1]{\ensuremath{\Ws(#1)}}
 \newcommand{\Winv}[1]{\ensuremath{\Ws^{-1}(#1)}}
 \newcommand{\Wr}[2]{\ensuremath{\Ws(#1; #2)}}


\newcommand{\G}{\mathcal{G}}
\newcommand{\B}{\mathcal{B}}
\newcommand{\E}{\mathcal{E}}
\newcommand{\cS}{\mathcal{S}}
\newcommand{\U}{\mathcal{U}}
\newcommand{\cL}{\mathcal{L}}
\newcommand{\N}{\mathcal{N}}
\newcommand{\cO}{\mathcal{O}}

\newcommand{\bN}{\mathbb{N}}
\newcommand{\bQ}{\mathbb{Q}}
\newcommand{\Exp}{\mathbb{E}}
\newcommand{\bC}{\mathbb{C}}
\newcommand{\bR}{\mathbb{R}}
\newcommand{\bZ}{\mathbb{Z}}
\newcommand{\bK}{\mathbb{K}}
\newcommand{\bF}{\mathbb{F}}

\newcommand{\fF}{\mathbf{F}}
\newcommand{\fA}{\mathbf{A}}
\newcommand{\Grp}{\mathbf{Grp}}
\newcommand{\Set}{\mathbf{Set}}
\newcommand{\Mor}{\mathrm{Mor}}
\newcommand{\Aut}{\mathrm{Aut}}
\newcommand{\Gal}{\mathrm{Gal}}
\newcommand{\Char}{\mathrm{char}}
\newcommand{\SL}{\mathrm{SL}}
\newcommand{\GL}{\mathrm{GL}}
\newcommand{\Span}{\mathrm{span}}
\newcommand{\Hom}{\mathrm{Hom}}
\newcommand{\Ext}{\mathrm{Ext}}
\newcommand{\im}{\mathrm{im}}
\newcommand{\col}{\mathrm{col}}

\newcommand{\cone}{\mathrm{cone}}
\newcommand{\rank}{\mathrm{rank}}
\newcommand{\RP}{\mathbb{R}\mathrm{P}}
\newcommand{\tr}{\mathrm{tr}}

\newcommand{\df}{\overset{\text{def}}{=}}
%\newcommand{\deg}{\mathrm{deg}}

\newcommand{\vx}{\mathbf{x}}
\newcommand{\vy}{\mathbf{y}}
\newcommand{\vu}{\mathbf{u}}
\newcommand{\vv}{\mathbf{v}}
\newcommand{\va}{\mathbf{a}}
\newcommand{\vb}{\mathbf{b}}
\newcommand{\vc}{\mathbf{c}}
\newcommand{\vd}{\mathbf{d}}
\newcommand{\vg}{\mathbf{g}}
\newcommand{\vk}{\mathbf{k}}
\newcommand{\vw}{\mathbf{w}}
\newcommand{\vz}{\mathbf{z}}
\newcommand{\vp}{\mathbf{p}}
\newcommand{\vq}{\mathbf{q}}
\newcommand{\vS}{\mathbf{S}}
\newcommand{\vf}{\mathbf{f}}
\newcommand{\bo}{\mathbf{1}}
\newcommand{\bz}{\mathbf{0}}
\newcommand{\id}{\mathrm{id}}

\newcolumntype{L}[1]{>{\raggedright\let\newline\\\arraybackslash\hspace{0pt}}m{#1}}
\newcolumntype{C}[1]{>{\centering\let\newline\\\arraybackslash\hspace{0pt}}m{#1}}

\newcommand\undermat[2]{%
  \makebox[0pt][l]{$\smash{\underbrace{\phantom{%
    \begin{matrix}#2\end{matrix}}}_{\text{$#1$}}}$}#2}
    
\newcommand\overmat[2]{%
  \makebox[0pt][l]{$\smash{\color{black}\overbrace{\phantom{%
    \begin{matrix}#2\end{matrix}}}^{\text{\color{black}#1}}}$}#2}

%%%%%%%%%%%%%%%%%%%%%%%%%%%%%%%%%%%%%%%%%%%%%%%%%%%%%%%%%%%%%%%%%%%%%%%%%%%%%
%% Begin of Document
%%%%%%%%%%%%%%%%%%%%%%%%%%%%%%%%%%%%%%%%%%%%%%%%%%%%%%%%%%%%%%%%%%%%%%%%%%%%%
\begin{document}
%%%%%%%%%%%%%%%%%%%%%%%%%%%%%%%%%%%%%%%%%%%%%%%%%%%%%%%%%%%%%%%%%%%%%%%%%%%%%
%% Here starts the poster
%%---------------------------------------------------------------------------
%% Format it to your taste with the options
%%%%%%%%%%%%%%%%%%%%%%%%%%%%%%%%%%%%%%%%%%%%%%%%%%%%%%%%%%%%%%%%%%%%%%%%%%%%%
\begin{poster}{
 % Show grid to help with alignment
 grid=false,
 % Column spacing
 colspacing=0.7em,
 % Color style
 headerColorOne=blue,
 headerFontColor=white,
 borderColor=blue,
 % Format of textbox
 textborder=faded,
 % Format of text header
 headerborder=open,
 headershape=roundedright,
 headershade=plain,
 background=none,
 bgColorOne=cyan!10!white,
 headerheight=0.11\textheight}
 % Eye Catcher
 {
      %\includegraphics[width=0.30\linewidth]{311_logo}
 }
 % Title
 {\sc\Huge Analyzing Boston 311 Requests with Gaussian Mixture Models}
 % Authors
 {Jane Huang, Isadora Nun, Weiwei Pan and Francisco Rivera\\
 {AM 207 Spring 2016}

}
 % University logo
 {
  \begin{tabular}{r}
    \includegraphics[height=0.11\textheight]{bos_logo_sm}
  \end{tabular}
 }
 

%%%%%%%%%%%%%%%%%%%%%%%%%%%%%%%%%%%%%%%%%%%%%%%%%%%%%%%%%%%%%%%%%%%%%%%%%%%%%%
%%% Now define the boxes that make up the poster
%%%---------------------------------------------------------------------------
%%% Each box has a name and can be placed absolutely or relatively.
%%% The only inconvenience is that you can only specify a relative position 
%%% towards an already declared box. So if you have a box attached to the 
%%% bottom, one to the top and a third one which should be inbetween, you 
%%% have to specify the top and bottom boxes before you specify the middle 
%%% box.
%%%%%%%%%%%%%%%%%%%%%%%%%%%%%%%%%%%%%%%%%%%%%%%%%%%%%%%%%%%%%%%%%%%%%%%%%%%%%%

%%%%%%%%%%%%%%%%%%%%%%%%%%%%%%%%%%%%%%%%%%%%%%%%%%%%%%%%%%%%%%%%%%%%%%%%%%%%%%
  \headerbox{Introduction: }{name=contribution,column=0,row=0,span=2}{
%%%%%%%%%%%%%%%%%%%%%%%%%%%%%%%%%%%%%%%%%%%%%%%%%%%%%%%%%%%%%%%%%%%%%%%%%%%%%%
Traditionally, Boston residents have been able to report problems such as potholes or graffiti through phone calls to city non-emergency services. Ubiquitous smartphone technology now also allows Boston residents to contact 311 services through the Citizens Connect App. 
\vskip0.2cm
\textbf{Goal:} Compare the effectiveness of responses to requests made through constituent calls and Citizens Connect App by modeling the geographic and response time distribution with Gaussian mixture models.
  \begin{center}
  \includegraphics[width=0.43\textwidth, height=0.20\textheight]{Intensity_call}
  \includegraphics[width=0.43\textwidth, height=0.20\textheight]{Intensity_app} \\
     \footnotesize{311 request volume intensity surface for constituent calls and Citizens Connect App (2015)}
  \end{center}
  }

%%%%%%%%%%%%%%%%%%%%%%%%%%%%%%%%%%%%%%%%%%%%%%%%%%%%%%%%%%%%%%%%%%%%%%%%%%%%%%
  \headerbox{The Model}{name=abstract,column=0,below=contribution}{
%%%%%%%%%%%%%%%%%%%%%%%%%%%%%%%%%%%%%%%%%%%%%%%%%%%%%%%%%%%%%%%%%%%%%%%%%%%%%%
    \begin{center}
    \includegraphics[width=0.60\textwidth, height=0.12\textheight]{graph_model}
    \end{center}
     \small{Our $K$-component Gaussian mixture model is defined as:}
\begin{align*}
\pi &\sim Dir(\alpha_0) &\tiny{(\text{Mixture Coefficient})}&\\
\Lambda_k &\sim Wish(W_0, \nu_0)& \tiny{(\text{Component Precision Matrix})}& \\
\mu_k | \Lambda_k &\sim \N(\eta_0, (\beta_0 \Lambda_k )^{-1}) &\tiny{(\text{Component Mean})}&\\
z_n | \pi &\sim \prod_{k=1}^K \pi_k^{z_{nk}} & \tiny{(\text{Label})}&\\
x_n | Z, \mu, \Lambda &\sim \prod_{k=1}^K \N(\mu_k, \Lambda_k)^{z_{nk}} &\tiny{(\text{Likelihood})}&
\end{align*}
Note that the set of hyper-parameters of our model is 
\begin{align*}
\theta = (\alpha_0, W_0, \nu_0, \eta_0, \beta_0).
\end{align*} }



 %%%%%%%%%%%%%%%%%%%%%%%%%%%%%%%%%%%%%%%%%%%%%%%%%%%%%%%%%%%%%%%%%%%%%%%%%%%%%%
   \headerbox{Results for Latent Clusters in 311 Data}{name=speed,column=2,row=0,span=2}{
 %%%%%%%%%%%%%%%%%%%%%%%%%%%%%%%%%%%%%%%%%%%%%%%%%%%%%%%%%%%%%%%%%%%%%%%%%%%%%%
 \setlength{\tabcolsep}{0pt}
 \textbf{Clustering based on MAP Estimates} Our different estimation techniques consistently indicate that for both the Citizens Connect and constituent call data, the largest cluster is characterized by a short response time, the next largest cluster is characterized by a somewhat longer response time, and the smallest cluster is characterized by a very slow mean response time. For example, for MAP, the results for constituent calls and the Citizens Connect App are similar, with the largest cluster centered at 4.5 days, the next largest at 50 days, and the smallest at 266 days. The covariances between geography and response time are larger for the slowest response time cluster compared to the faster response time clusters. 
 \begin{tabular}{cc}
   \includegraphics[width=0.23\textwidth, height=0.15\textheight]{Call_3D}      
   \includegraphics[width=0.25\textwidth, height=0.15\textheight]{Call_2D}
    &
   \includegraphics[width=0.23\textwidth, height=0.15\textheight]{App_3D}   
   \includegraphics[width=0.25\textwidth, height=0.15\textheight]{App_2D}\\
   \footnotesize{ \textbf{(a)} 3D \& 2D projections of clusters (constituent calls)}&
   \footnotesize{ \textbf{(b)} 3D \& 2D projections of clusters (Citizens Connect App)}
  \end{tabular}
  
  \setlength{\tabcolsep}{0pt}
 \begin{tabular}{L{8cm}C{8cm}}
   \textbf{Posterior predictive distribution} We show the posterior predictive distributions generated from our Gibbs sampler. While the full complexity of the Boston 311 data is not completely described by the three component Gaussian mixture model, a mixture model is still useful for understanding how the 311 requests spread apart into components differentiated by response times. 
    &
   \includegraphics[width=0.46\textwidth, height=0.15\textheight]{post_pred}   
   \footnotesize{ \textbf{(b)} Posterior predictive distributions for call and app data}
  \end{tabular}

   }

 %%%%%%%%%%%%%%%%%%%%%%%%%%%%%%%%%%%%%%%%%%%%%%%%%%%%%%%%%%%%%%%%%%%%%%%%%%%%%%

   \headerbox{Component Analysis}{name=tracking,column=2,span=2,below=speed,above=bottom}{
 %%%%%%%%%%%%%%%%%%%%%%%%%%%%%%%%%%%%%%%%%%%%%%%%%%%%%%%%%%%%%%%%%%%%%%%%%%%%%%
 {

   }

 \begin{tabular}{cc}
   \includegraphics[width=0.44\textwidth, height=0.19\textheight]{Call_cluster_profiles}&
   \includegraphics[width=0.44\textwidth, height=0.19\textheight]{App_cluster_profiles}\\
   \footnotesize{\textbf{(a)} Cluster profiles (constituent call)} &  \footnotesize{\textbf{(a)} Cluster profiles (Citizens Connect App)}
  \end{tabular}
  \vskip0.2cm
 After identifying the clusters, we examined the qualitative data associated with each request to understand the composition of each cluster by type of service requested. While for both Citizens Connect App and constituent call requests, the clustering is comparable in terms of slow, medium, and fast response times, the differing compositions between the two sets of clusters hints that the reasons for the lagging response times may be associated with different latent characteristics. These results provide a foundation for building more detailed models examining predictors of lagging response times. 
   }


 %%%%%%%%%%%%%%%%%%%%%%%%%%%%%%%%%%%%%%%%%%%%%%%%%%%%%%%%%%%%%%%%%%%%%%%%%%%%%%%
  \headerbox{Methods}{name=algorithm,column=1,above=bottom,below=contribution}{
 %%%%%%%%%%%%%%%%%%%%%%%%%%%%%%%%%%%%%%%%%%%%%%%%%%%%%%%%%%%%%%%%%%%%%%%%%%%%%%%
  We analyze a randomly drawn subset of 2015 reports made to Boston 311 through either constituent calls or the Citizens Connect App. 
 \vskip0.2cm
  This dataset is publicly available at:
  \begin{align*}
   \small{\mathtt{https://data.cityofboston.gov/}}. 
   \end{align*}
  \vskip0.2cm
Each datapoint, $\vx_n$, is a three-dimensional vector with features: 
\begin{center}
(\emph{response time}, \emph{longitude}, \emph{latitude}).
\end{center}
  \vskip0.2cm
Using the Bayes Information Criterion, we chose a model with $K=3$ components. 
 \vskip0.2cm
We compute both maximum likelihood estimates and maximum a posteriori estimates for the component parameters through expectation maximization. We compared these point estimates to the results for simulated annealing. 
 \vskip0.2cm
The point estimates are used to initialize a Gibbs sampler to obtain the posterior distribution. These samples are then used to generate the posterior predictive distribution to compare to the original data.
   }
\end{poster}%
%
\end{document}
